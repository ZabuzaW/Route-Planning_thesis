% Conclusion
\chapter{Conclusion}\label{conclusion}
	Route planning is a problem of much interest that gained a lot of interest in the last decades. Problem settings like \uniModal route planning are
	well researched, efficient solutions were developed. Corresponding research is now focused on \multiModal routing and other difficult problems
	occurring in practice, such as turn penalties and multi criteria routing.

% Future Work
\section{Future Work}
Blabla. \todo{Write something...}

% Summary
\section{Summary}
	We have presented common and established models for road and transit networks. Graph based solutions are straightforward representations of the network,
	but can not easily adapt to time dependent data, such as transit networks. Timetables are non-graph based alternatives for public transit networks
	which fit their structure better than static graphs. Additionally, a link graph can be used to combine graph based models for multiple networks in a
	straightforward manner. While it might not necessarily be an effective approach, it makes route planning on combined networks for graph based
	algorithms possible.\\\\
	In order to explain more sophisticated route planning approaches, we presented the \nearestNeighborProblem and thoroughly discussed an
	efficient solution to the problem and various variants, using {\coverTree}s.\\\\
	We covered basic route planning algorithms, such as \dijkstra and common optimizations like \astar. The effectiveness of \astar heavily relies on the chosen
	heuristic, which depends on the underlying structure of the network. \alt was presented as a solution to this problem, providing a general applicable heuristic
	which is based on the actual shortest path distances to chosen landmarks. For an overview of more sophisticated \uniModal time-independent
	algorithms, we refer to \libref{routePlanningOverview}.
	
	\csa was introduced as an efficient approach for time-dependent route planning on timetables. The approach is very simple, it just processes all connections
	available after the initial departure time. \csa is fast because it heavily exploits cache locality \libref{cacheLocality} and other low-level optimizations
	for arrays.
	
	For \multiModal route planning we showed how \dijkstra can be adapted to run on a link graph, representing a combined network. Further, we presented the
	general concept of \anr and proposed a simplified variant of it, generalized to an arbitrary algorithm for road networks and another algorithm for transit
	networks. This makes it possible to combine a graph based solution like \dijkstra, or even more sophisticated approaches, for the road network,
	with a timetable based approach for transit networks, such as \csa.\\\\
	Further, we presented experimental results of implementations in the \cobweb project \libref{cobweb} and discussed them.
	For the experiments three data sets are used, \freiburgR, \stuttgartR and \switzerlandR.
	The setup, as well as the structure of the input data, was thoroughly explained. {\coverTree}s and \csa turned out to be a a very
	efficient solution to their corresponding problems. \dijkstra works well for short range queries,
	but scales bad for increasing ranges. Further, it lacks behind more sophisticated approaches as seen in \figref{uniModalTimeIndependentResultsExternalOverview}.
	\astar using $\asTheCrowFlies$ does not perform well on networks used for road planning. While \alt, if carefully implemented, typically beats \dijkstra,
	especially for mid to long range queries. In practice, link graphs are often not feasible due to extreme demands on space capacity. For \multiModal routing
	\dijkstra performs similar to \uniModal routing, being feasible for short range queries, but scaling bad for increasing ranges. \anr, if paired with efficient algorithms
	for both networks, is a promising approach to \multiModal route planning, as seen in \libref{accessNodeRouting}.\\\\
	Route planning, in particular in practice, is a complex topic. A typical application needs to account for more than just finding a route with the shortest travel time.
	Turn penalties and multi-criteria routing, such as the cost of a trip, are important factors for a client and need to be considered. A similar observation is done
	for \multiModal routing, where transportation mode restrictions, in practice, are not just a set of available modes, but rather a complex model with
	multiple states depending on previous states, as explained in \sectionref{multiModal_sec}.
	
	Most algorithms do not adapt well to such restrictions, leading to the development of many very specialized solutions. Because of that, existing
	approaches, such as \anr, rather try to combine multiple algorithms, all suited well for their own specialized type of network. In particular
	for \multiModal routing, including common restrictions occurring in practice, there does not yet exist a feasible solution for networks of a large scale,
	such as big countries or even continents. However, with increasing research in the last decade, many promising approaches were developed and
	a solution does not seem too far.