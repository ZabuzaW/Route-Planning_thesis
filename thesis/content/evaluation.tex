% Evaluation
\section{Evaluation}\label{evaluation}
	In this section we report on our experimental results for the presented algorithms on three data sets of increasing size.
	Therefore, we first give insights on the data sets and how the network models are obtained. Afterwards we evaluate
	{\coverTree}s, \dijkstra, \astar (with $\asTheCrowFlies$), \alt, \csa and \multiModal methods such as the adopted \dijkstra
	and our simplified version of \anr on the given data sets.\\\\
	When evaluating shortest path queries on randomly chosen source and target nodes, the resulting paths tend to be long-range.
	However, in practice most queries are only local and algorithms like \dijkstra do not scale well with increasing range.
	To overcome this measurement problem, we introduce the notion of a \textit{Dijkstra rank} \libref{dijkstraRank}.
	\begin{mydef}\label{dijkstraRank}
		Given a graph $G = (V, E)$, the \textnormal{Dijkstra rank} of a node $v \in V$ is the number of the iteration in which,
		when running \dijkstra on the graph, it is polled from the priority queue (see \textbf{line 7} of \algoref{dijkstra}).
		
		That is the position $i$ for $v_i$ in the order of vertices when sorted ascending by their distance to the source, i.e.
		\begin{align*}
			v_1, v_2, \ldots, v_{|V|}
		\end{align*}
		with $\dist(v_i) \le \dist(v_{i + 1})$ for all $i$.
	\end{mydef}\quad\\
	Instead of choosing queries randomly, we only choose source nodes randomly and then select targets by their
	\textit{Dijkstra rank} to the source. Queries can then be sorted by the \textit{Dijkstra rank} and, by that,
	evaluated in terms of increasing range.

%Input data
\subsection{Input data}
	We consider three data sets, consisting of road and public transit data. The road network is extracted from \osm \libref{osm}
	formatted data and transit data is given in the \gtfs \libref{gtfs} format.\\\\
	Our data sets represent the region around the german cities \freiburgR and \stuttgartR. Their road network is
	of similar size, while our transit data for \freiburgR only includes tram data, whereas the data for \stuttgartR
	also includes train and bus connections. The size of our transit network for \stuttgartR is about ten times the size of
	the network for \freiburgR.
	
	Furthermore, we include a road and transit network for the country \switzerlandR. The transit data consists of train, tram and
	bus connections. Both networks are about three times the size of the {\stuttgartR}s.\\\\
	We obtain our road networks from \libref{freiburgRoadSource, stuttgartRoadSource, switzerlandRoadSource}
	and our transit networks from \libref{freiburgTransitSource, switzerlandTransitSource}.
	The transit data used for \stuttgartR is under restricted public access (refer to \libref{vvsContact}).

%OSM
\subsubsection{\osm}
	\osm \libref{osm} (OpenStreetMap) data is represented in a \xml structure describing
	\begin{itemize}
		\item[1.] \textit{nodes}, with an unique identifier and a coordinate given as pair of latitude and longitude;
		\item[2.] \textit{ways}, also with an unique identifier, consisting of multiple nodes referenced by their identifier;
		\item[3.] \textit{relations}, consisting of nodes, ways and other relations, representing relationships between the referenced data;
		\item[4.] \textit{tags} as key-value pairs, storing metadata about the other items.
	\end{itemize}
	\begin{lstlisting}[caption={\osm example data set, derived from \libref{osmExample}.},label={osmExample},style={XMLStyle},mathescape={true},
		float,floatplacement=ht]
<?xml version='1.0' encoding='UTF-8'?>
<osm version="0.6">
  <bounds minlon="7.253190" minlat="47.299090" maxlon="9.246965" maxlat="48.751520"/>
  <node id="29764598" lat="47.8512831" lon="7.9230269"/>
  <node id="669209525" lat="47.8513215" lon="7.9231227"/>
  <node id="3993821274" lat="47.8513342" lon="7.923183"/>
  <node id="832450227" lat="47.8157938" lon="8.8487527">
    <tag k="highway" v="motorway_junction"/>
    <tag k="name" v="Kreuz Hegau"/>
  </node>
  <node id="100036455" lat="47.5728421" lon="8.0365409">
    <tag k="name" v="Niederhof"/>
    <tag k="traffic_sign" v="city_limit"/>
  </node>
  <way id="29764598">
    <nd ref="669209525"/>
    <nd ref="3993821274"/>
    <tag k="highway" v="motorway"/>
    <tag k="oneway" v="yes"/>
  </way>
  <relation id="56688">
    <member type="node" ref="29764598" role=""/>
    <member type="node" ref="669209525" role=""/>
    <member type="way" ref="29764598" role=""/>
    <tag k="name" v="Bus line 1"/>
    <tag k="network" v="VVW"/>
    <tag k="ref" v="1"/>
    <tag k="route" v="bus"/>
    <tag k="type" v="route"/>
  </relation>
</osm>
	\end{lstlisting}\quad\\
	A small \osm example data set is shown by \lstref{osmExample}. Ways are used to represent roads consisting
	of nodes. Tags are used to describe metadata like speed limits for a road or whether it is a one-way street or not.
	However, the format also contains a lot of data not directly relevant for route planning, like shapes of buildings
	and outlines of public parks. Therefore, we filter \osm data and only keep relevant information.
	\begin{lstlisting}[caption={Tag filter for \osm ways.},label={osmFilter},style={FilterStyle},mathescape={true},
		float,floatplacement=ht]
--KEEP

#highways
highway=motorway
highway=trunk
highway=primary
highway=secondary
highway=tertiary
highway=residential
highway=living_street
highway=unclassified
highway=cycleway

#highwaylinks
highway=motorway_link
highway=trunk_link
highway=primary_link
highway=secondary_link
highway=tertiary_link
highway=residential_link

#non-standard
way=primary
way=seconday

--DROP

area=yes
train=yes
access=no
type=multipolygon
railway=platform
railway=station
highway=proposed
highway=construction
building=yes
building=train_station
	\end{lstlisting}\quad\\\\
	As we are only interested in the road network itself, we start by reading the ways. We filter them based on the tags
	described by \lstref{osmFilter}. Ways having at least one of the key-value pairs described under \textit{$--$KEEP}
	and none of the pairs under \textit{$--$DROP} are kept, as they represent roads of the network. All other ways are
	rejected, as well as all relations.
	After that, we read the nodes and only keep nodes that occurred at least once in any of the ways that passed the filter.
	Our road network is then build using the remaining nodes as graph nodes, translating the ways into edges between the nodes.
	
	Ways with a positive \textit{oneway} tag are translated into edges only going into the given direction, else we generate
	both edges for both directions. The cost of an edge is computed as the time it needs to travel the direct distance between
	the source and destination coordinates (see \defref{asTheCrowFlies}) with a certain speed. The speed is determined either
	by a given \textit{maxspeed} tag or the average speed for the road type defined by the \textit{highway} tag.
	Therefore, we use the average speed references shown by \tableref{highwaySpeeds}.
	\begin{table}[ht]
	 	\begin{center}
	 		\phantom{v}\quad\\
	 		\begin{tabular}{|l|r|}
	 			\hline
	 			\multicolumn{1}{|c|}{tag value}	&\multicolumn{1}{c|}{$\o$ km/h}\\\hline

				motorway		&$120$\\
				trunk			&$110$\\
				primary		&$100$\\
				secondary		&$80$\\
				tertiary		&$70$\\
				motorway\_link	&$50$\\
				trunk\_link		&$50$\\
				primary\_link		&$50$\\
				secondary\_link	&$50$\\
				residential		&$50$\\
				unclassified		&$40$\\
				unsurfaced		&$30$\\
				road			&$20$\\
				cycleway		&$14$\\
				living\_street		&$7$\\
				service		&$7$\\\hline
			\end{tabular}
		\end{center}
		\caption{Average speed in km/h for a \osm way with the corresponding value for the \textit{highway} tag.}
		\label{highwaySpeeds}
	\end{table}
	\begin{table}[ht]
	 	\begin{center}
	 		\phantom{v}\quad\\
			\begin{tabular}{|l||r|r|r|r|}
				\hline
							&\multicolumn{2}{c|}{data (MB)}	&\multicolumn{2}{c|}{Road graph}\\
							&\multicolumn{1}{c|}{raw}	&\multicolumn{1}{c|}{filtered}	&\multicolumn{1}{c|}{nodes}
								&\multicolumn{1}{c|}{edges}\\\hline
				\freiburgR		&$2\,260$	&$86$		&$743\,003$		&$1\,494\,883$\\
				\stuttgartR		&$2\,420$	&$118$	&$973\,142$		&$1\,950\,978$\\
				\switzerlandR	&$5\,530$	&$279$	&$2\,627\,645$	&$5\,226\,060$\\\hline
			\end{tabular}
		\end{center}
		\caption{Size of the \osm data sets, in megabyte (MB) before and after filtering, and the size of the resulting road graphs
			in amount of nodes $|V|$ and edges $|E|$.}
		\label{osmSize}
	\end{table}\quad\\\\
	The size of the resulting road graphs (see \sectionref{roadGraphSec}) for all three data sets is reported in \tableref{osmSize}.
	As seen, filtering the \osm data sets beforehand reduces the size of data that is to be processed by $95\%$ to $97\%$.
	The road graphs have approximately two edges per node. This is due to most streets being two-way streets, thus generating
	two edges per connection between two nodes. Obviously, road junctions are, compared to the amount of nodes, rare and thus,
	multiple edges do only rarely share the same node. The in- and outdegree of nodes is extremely low, mostly $2$ ($\approx 80\%$),
	as seen in \tableref{roadDegree}.
	\begin{table}[ht]
	 	\begin{center}
	 		\phantom{v}\quad\\
	 		\begin{tabular}{l}
			\begin{tabular}{|l||r|r|r|r|r|r|r|}
				\hline
							&\multicolumn{7}{c|}{indegree $\degG^{-}$}\\
							&\multicolumn{1}{c|}{0}	&\multicolumn{1}{c|}{1}	&\multicolumn{1}{c|}{2}	&\multicolumn{1}{c|}{3}
								&\multicolumn{1}{c|}{4}	&\multicolumn{1}{c|}{5}	&\multicolumn{1}{c|}{6}\\\hline
				\freiburgR		&$90$		&$64\,990$		&$611\,055$		&$59\,751$		&$7\,057$	&$58$		&$2$\\
				\stuttgartR		&$145$	&$109\,808$		&$759\,157$		&$93\,354$		&$10\,599$	&$76$		&$3$\\
				\switzerlandR	&$325$	&$235\,069$		&$2\,201\,945$	&$174\,333$		&$15\,767$	&$202$	&$4$\\\hline
			\end{tabular}\\
			\quad\\
			\quad\\
			\begin{tabular}{|l||r|r|r|r|r|r|r|r|}
				\hline
							&\multicolumn{8}{c|}{outdegree $\degG^{+}$}\\
							&\multicolumn{1}{c|}{0}	&\multicolumn{1}{c|}{1}	&\multicolumn{1}{c|}{2}	&\multicolumn{1}{c|}{3}
								&\multicolumn{1}{c|}{4}	&\multicolumn{1}{c|}{5}	&\multicolumn{1}{c|}{6}	&\multicolumn{1}{c|}{7}\\\hline
				\freiburgR		&$105$	&$65\,336$		&$610\,353$		&$60\,059$		&$7\,088$	&$60$		&$2$	&$0$\\
				\stuttgartR		&$162$	&$110\,002$		&$758\,740$		&$93\,545$		&$10\,607$	&$83$		&$3$	&$0$\\
				\switzerlandR	&$328$	&$235\,255$		&$2\,201\,711$	&$174\,247$		&$15\,884$	&$215$	&$4$	&$1$\\\hline
			\end{tabular}
			\end{tabular}
		\end{center}
		\caption{Table showing the number of nodes of the corresponding road graph that have a certain in- or outdegree.
		That is, the number of ingoing and outgoing edges respectively.}
		\label{roadDegree}
	\end{table}

%GTFS
\subsubsection{\gtfs}
	\gtfs \libref{gtfs} is short for General Transit Feed Specification, it defines a common format
	for public transit schedules. It comes compressed as \zip archive, consisting multiple
	text files formatted as \csv tables. The mandatory tables are
	\begin{itemize}
		\item[1.] \textit{agency.txt}, defining metadata about the transit agency;
		\item[2.] \textit{routes.txt}, containing information about complete routes, like all trips belonging to a bus line;
		\item[3.] \textit{trips.txt}, consisting of single trips, belonging to a route;
		\item[4.] \textit{stop\_times.txt}, having departure and arrival times at the stops for all connections in the network;
		\item[5.] \textit{stops.txt}, providing metadata and coordinates of all stops;
		\item[6.] \textit{calendar.txt} defining the service pattern at which routes are available.
	\end{itemize}
	Furthermore, there are a couple of optional tables of which we are only interested in
	\begin{itemize}
		\item[7.] \textit{transfer.txt}, provides transfer possibilities between stops and their duration.
	\end{itemize}
		\begin{lstlisting}[caption={\gtfs example data set, inspired by \libref{gtfsExample}.},label={gtfsExample},style={GTFSStyle},mathescape={true},
		float,floatplacement=ht]
// agency.txt
agency_id, agency_name, agency_url, agency_timezone, agency_phone, agency_lang
FunBus, The Fun Bus, , (310) 555-0222, en

// routes.txt
route_id, route_short_name, route_long_name, route_desc, route_type
A, 17, Mission, From lower Mission to Downtown., 3

// trips.txt
route_id, service_id, trip_id, trip_headsign, block_id
A, WE, AWE1, Downtown, 1
A, WE, AWE2, Downtown, 2

// stop_times.txt
trip_id, arrival_time, departure_time, stop_id, stop_sequence, pickup_type, drop_off_type
AWE1, 0:06:10, 0:06:10, S1, 1, 0, 0
AWE1, 0:06:20, 0:06:30, S3, 3, 0, 0
AWE1, 0:06:45, 0:06:45, S6, 5, 0, 0
AWD1, 0:06:10, 0:06:10, S1, 1, 0, 0
AWD1, 0:06:20, 0:06:20, S3, 3, 0, 0
AWD1, 0:06:45, 0:06:45, S6, 6, 0, 0

// stops.txt
stop_id, stop_name, stop_desc, stop_lat, stop_lon, stop_url, location_type, parent_station
S1, Mission St. & Silver Ave., , 37.728631, -122.431282, , ,
S3, Mission St. & 24th St., , 37.75223, -122.418581, , ,
S6, Mission St. & 15th St., , 37.766629, -122.419782, , ,

// calendar.txt
service_id, monday, tuesday, wednesday, thursday, friday, saturday, sunday, start_date, end_date
WE, 0, 0, 0, 0, 0, 1, 1, 20060701, 20060731
WD, 1, 1, 1, 1, 1, 0, 0, 20060701, 20060731

// transfers.txt
from_stop_id, to_stop_id, transfer_type, min_transfer_time
S3, S6, 2, 300
S6, S3 3, 180
	\end{lstlisting}\quad\\
	An example feed can be seen in \lstref{gtfsExample}. The format is similar to our definition of timetables (see \sectionref{timetable_sec}),
	with the difference that connections are not directly given as edges departing from one stop and another, but as pair of arrival and
	departure time at stops. Also, it contains a lot of metadata which we do not process.\\\\
	Construction of a realistic time expanded transit graph (see \defref{realisticTransitGraph}) is straightforward and mainly
	involves around parsing \textit{stop\_times.txt}. We build two nodes for every entry, one representing the arrival event at the stop
	and another for the departure. Furthermore, we create a transfer node for every arrival node, indicating a transfer at the given stop.
	Each arrival node is then connected by an edge with its corresponding departure and transfer node.
	
	After parsing all data, we connect departure nodes with the arrival nodes at the next stop in a trip. Therefore, we
	process each trip and follow the \textit{stop\_times.txt} entries belonging to that trip in the order defined by
	the \textit{stop\_sequence} field.
	
	As a next step, waiting edges are created by sorting transfer nodes of a stop ascending in time and then creating edges
	connecting them in that order. Finally, every departure node is connected to its previous transfer node. We find the
	transfer node by using a \textit{binary search} \libref{binarySearch} on the sorted list of transfer nodes for this stop.\\\\
	Timetables (see \defref{timetable}) are received similar. But simpler, as transfer nodes are not present. We process
	all stops and trips defined in \textit{stops.txt} and \textit{trips.txt} and obtain the sets $S$ and $T$ respectively.
	Connections are created by again processing entries in \textit{stop\_times.txt}, belonging to one trip, in the sequence
	defined by \textit{stop\_sequence}. We create one connection for every departure node with the corresponding
	next arrival node.
	
	For the footpaths, we initially take the transfers given in \textit{transfers.txt}. In order to increase the quality
	of our footpath model, we also connect stops with footpaths if they are within $600$ meters to each other.
	
	However, our footpaths need to fulfill strong properties (see \sectionref{timetable_sec}), which the given
	transfers usually not obey. Therefore, we have to add self-loop footpaths, if not present. And we need to compute
	the transitive closure of the given footpaths in order to ensure that they are transitively closed. Therefore, it is
	crucial that the range, for which close stops are connected, is kept low. Else, the amount of footpaths dramatically increases
	due to the transitive closure.
	
	The \textit{triangle inequality} property is ensured by rejecting given transfer durations and approximating all durations by using $\asTheCrowFlies$.
	Additionally, all footpath durations must not be lower than the transfer buffer used for the self-loop footpaths. We do so by taking the
	$\max$ of the transfer buffer and the calculated duration.
	\begin{table}[ht]
	 	\begin{center}
	 		\phantom{v}\quad\\
	 		\begin{tabular}{l}
			\begin{tabular}{|l||r|r|r|}
				\hline
							&\multicolumn{1}{c|}{data (KB)}	&\multicolumn{2}{c|}{Transit graph}\\
							&	&\multicolumn{1}{c|}{nodes}	&\multicolumn{1}{c|}{edges}\\\hline
				\freiburgR		&$1\,713$		&$613\,329$		&$1\,006\,862$\\
				\stuttgartR		&$32\,213$		&$4\,517\,511$	&$7\,415\,894$\\
				\switzerlandR	&$75\,477$		&$32\,688\,498$	&$53\,370\,236$\\\hline
			\end{tabular}\\
			\quad\\
			\begin{tabular}{|l||r|r|r|r|}
				\hline
							&\multicolumn{4}{c|}{Timetable}\\
							&\multicolumn{1}{c|}{stops}	&\multicolumn{1}{c|}{trips}
								&\multicolumn{1}{c|}{connections} &\multicolumn{1}{c|}{footpaths}\\\hline
				\freiburgR		&$713$	&$13\,249$		&$191\,194$		&$255\,495$\\
				\stuttgartR		&$7\,877$	&$90\,475$		&$1\,415\,362$	&$1\,926\,611$\\
				\switzerlandR	&$30\,227$	&$1\,014\,699$	&$9\,881\,467$	&$3\,793\,581$\\\hline
			\end{tabular}\\
			\quad\\
			\begin{tabular}{|l||r|r|r|r|}
				\hline
							&\multicolumn{4}{c|}{Footpaths}\\
							&\multicolumn{1}{c|}{given}	&\multicolumn{1}{c|}{self-loops}
								&\multicolumn{1}{c|}{close} &\multicolumn{1}{c|}{closure}\\\hline
				\freiburgR		&$0$		&$713$		&$9\,008$		&$245\,774$\\
				\stuttgartR		&$6\,080$	&$7\,877$		&$73\,730$		&$1\,838\,924$\\
				\switzerlandR	&$22\,402$	&$30\,227$		&$174\,698$		&$3\,566\,254$\\\hline
			\end{tabular}
			\end{tabular}
		\end{center}
		\caption{Size of the \gtfs feeds, in kilobyte (KB) and the size of the resulting realistic time expanded transit graphs
			in amount of nodes $|V|$ and edges $|E|$. Also, the size of the obtained timetable and details about the
			footpath generation.\\
			The column \textit{given} denotes the amount of footpaths already given in the \textit{transfer.txt} file,
			\textit{self-loops} represents how many missing self-loop paths were added. Likewise does \textit{close} report how
			many footpaths we added for connecting close stops with each other. And \textit{closure} denotes
			the amount of paths added to ensure that the model is transitively closed.}
		\label{gtfsSize}
	\end{table}\quad\\\\
	\tableref{gtfsSize} reports the size of the feed and the resulting network. It can be clearly seen that a timetable has a
	much smaller amount of objects, compared to a realistic time expanded transit graph. In particular compared to the size of a
	road graph (see \tableref{osmSize}). This even becomes worse if we use it to construct a link graph, as seen in \sectionref{linkGraph_sec},
	as we need to add an incoming and outgoing edge for each arrival node, in order to connect it with the road graph. \tableref{linkGraphSize}
	reports the exact amount of added link edges.
	\begin{table}[ht]
	 	\begin{center}
	 		\phantom{v}\quad\\
			\begin{tabular}{|l||r|}
				\hline
							&\multicolumn{1}{c|}{link edges}\\\hline
				\freiburgR		&$306\,906$\\
				\stuttgartR		&$1\,944\,388$\\
				\switzerlandR	&$19\,584\,786$\\\hline
			\end{tabular}
		\end{center}
		\caption{Amount of link edges that are added when combining road with transit graphs to create a link graph.}
		\label{linkGraphSize}
	\end{table}\quad\\\\

%Experiments
\subsection{Experiments}
	Blabla

%Nearest neighbor computation
\subsubsection{Nearest neighbor computation}
	Blabla

%Uni-modal routing
\subsubsection{Uni-modal routing}
	Blabla

%Multi-modal routing
\subsubsection{Multi-modal routing}
	Blabla

%Summary
\subsection{Summary}
	Blabla